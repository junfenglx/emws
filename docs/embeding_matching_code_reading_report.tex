\documentclass[]{article}
\usepackage{nips15submit_e,times}

\usepackage[colorlinks=true]{hyperref}
\usepackage{url}


% for math
\usepackage{mathtools}

% for pictures
\usepackage{graphicx}

% datetime
\usepackage{datetime}


% chinese fonts
\usepackage{my_arch_ctex_fonts}

% for code
\usepackage{minted}


%opening
\title{Embedding Matching源码阅读报告}


\author{
	Junfeng~Hu\thanks{ \url{http://junfenglx.github.io/}} \\
	15210240075 \\
	School of Computer Science\\
	Fudan University\\
	No.825, Zhangheng Road, Shanghai \\
	\texttt{15210240075@fudan.edu.cn} \\
}

\nipsfinalcopy % Uncomment for camera-ready version

\begin{document}

\maketitle

% \begin{abstract}

% \end{abstract}

\tableofcontents

\section{代码概况}
《Accurate Linear-Time Chinese Word Segmentation via Embedding Matching》
论文代码基于Python实现。

code文件夹是代码实现,working\_data文件夹包含PKU数据集,以及perl score脚本。

在code文件夹下一共有以下几个文件:
\begin{itemize}
\item config.txt 实验配置文件
\item predict.py 使用已有模型对预料库进行分词
\item script2.py 训练模型脚本,重现实验
\item seg2.py 算法实现文件
\item word2vec2.py gensim的word2vec实现的修改版本。
\end{itemize}

该代码依赖环境如下:
\begin{itemize}
\item Linux like OS
\item Python 2.7, not Python 3.X
\item Numpy >= 1.9
\item gensim == 0.10.3
\end{itemize}

\section{实验重现}
在code文件夹下运行命令: \mint{bash}|python script2.py config.txt|
训练过程大概需要1个小时左右。config.txt文件中指定迭代10次,当然也可以增大
和减小迭代次数。

最终对测试集分词的结果如表~\ref{tab:pku_experiment_result}
\begin{table}[t]
\caption{PKU数据集实验重现结果}
\label{tab:pku_experiment_result}
\begin{center}
\begin{tabular}{ll}
\multicolumn{1}{c}{\bf  指标} & \multicolumn{1}{c}{\bf 大小} \\ \hline
F MEASURE &0.950 \\
OOV Rate  &0.058 \\
OOV Recall Rate    &0.756 \\
IV Recall Rate &0.957 \\
\end{tabular}
\end{center}
\end{table}

和其论文中得出的结果基本一致。

\section{源码分析}

code文件夹下seg2.py是包含论文算法实现代码的文件。

其主要包含类Seger,该类是算法实现类,其继承gensim的Word2Vec类。

该类主要方法见图~\ref{fig:seger_diagram}

\begin{figure}
\centering
\includegraphics[width=1.0\linewidth]{../seger_diagram}
\caption{Seger class diagram}
\label{fig:seger_diagram}
\end{figure}

算法相关的函数主要有三个:
\begin{enumerate}
\item predict\_single\_position
\item train\_gold\_per\_sentence
\item predict\_sentence\_greedy
\end{enumerate}

\renewcommand\listingscaption{Code}

\subsection{predict\_single\_position}

predict\_single\_position函数的python doc见Code~\ref{code:predict_single_position}
\begin{listing}[H]
\begin{minted}[frame=single, label=predict\_single\_position]{python}
    def predict_single_position(self, sent, pos,
                                prev2_label,
                                prev_label, states=None):
        """
        predict a character's label
        :param sent: the sentence
        :param pos: the character position
        :param prev2_label: second previous label
        :param prev_label: first previous label
        :param states: the previous iter states vector
        :return: softmax_score2, feature_index_list,
         pred_index_list2, feature_vec, pred_matrix2
        """
\end{minted}
\caption{predict\_single\_position方法}
\label{code:predict_single_position}
\end{listing}

predict\_single\_position主要代码片段见Code~\ref{code:predict_single_position关键代码}
\begin{listing}[H]
\begin{minted}[frame=single, label=predict\_single\_position关键代码]{python}
pred_tuple = tuple([self.su_prefix + varient + u
    for varient in self.state_varient])
if pred_tuple[0] in self.vocab and pred_tuple[1] in self.vocab:
    pass
else:
    pred_tuple = None
    if self.train_mode:
        print 'Unknown candidate! Should NOT happen during training!'
        assert False

# constant $LABEL0, $LABEL1
# for unknown words in test
pred_tuple2 = tuple([self.label0_as_vocab, self.label1_as_vocab])

softmax_score = None
if pred_tuple:
    pred_index_list = [self.vocab[pred].index for pred in pred_tuple]
    pred_matrix = self.syn1neg[pred_index_list]

    if block is not None:
        pred_matrix = multiply(block, pred_matrix)
    elif self.drop_out:
        pred_matrix = (1 - self.dropout_rate) * pred_matrix

    raw_score = exp(dot(feature_vec, pred_matrix.T))
    softmax_score = raw_score / sum(raw_score)

pred_index_list2 = [self.vocab[pred].index for pred in pred_tuple2]
pred_matrix2 = self.syn1neg[pred_index_list2]

if block is not None:
    pred_matrix2 = multiply(block, pred_matrix2)
elif self.drop_out:
    # should be pred_matrix2
    pred_matrix2 = (1 - self.dropout_rate) * pred_matrix2

raw_score2 = exp(dot(feature_vec, pred_matrix2.T))
softmax_score2 = raw_score2 / sum(raw_score2)
# print pred_matrix2.shape, pred_matrix.shape
if pred_tuple:
    softmax_score2 = np_append(softmax_score2, softmax_score)
    pred_index_list2.extend(pred_index_list)
    pred_matrix2 = np_append(pred_matrix2, pred_matrix, axis=0)
\end{minted}
\caption{predict\_single\_position关键代码}
\label{code:predict_single_position关键代码}
\end{listing}

\subsection{train\_gold\_per\_sentence}

train\_gold\_per\_sentence函数的python doc见Code~\ref{code:train_gold_per_sentence}
\begin{listing}[H]
\begin{minted}[frame=single, label=train\_gold\_per\_sentence]{python}
    def train_gold_per_sentence(self, sentence, alpha, work=None):
        """
        :param sentence: the segmented sentence
        :param alpha: the learning rate
        :param work: self.layer1_size vector,
         initialized with zero, not use
        :return: words count_sum, train error_sum for report
        """
\end{minted}
\caption{train\_gold\_per\_sentence方法}
\label{code:train_gold_per_sentence}
\end{listing}

train\_gold\_per\_sentence主要代码片段见Code~\ref{code:train_gold_per_sentence关键代码}
\begin{listing}[H]
\begin{minted}[frame=single, label=train\_gold\_per\_sentence关键代码]{python}
prev2_label, prev_label = 0, 0
for pos in range(count_sum):

    softmax_score, feature_index_list, pred_index_list,\
    feature_vec, pred_matrix = self.predict_single_position(
            sentence, pos, prev2_label, prev_label,
            states=label_list)

    true_label = label_list[pos]
    if true_label == 0:
        gold_score = [1.0, 0.0, 1.0, 0.0]
    elif true_label == 1:
        gold_score = [0.0, 1.0, 0.0, 1.0]

    error_array = gold_score - softmax_score
    error_sum += sum(abs(error_array)) / len(error_array)
    gb = error_array * alpha
    neu1e = zeros(self.non_fixed_param)
    neu1e += dot(gb, pred_matrix[:, 0:self.non_fixed_param])

    if self.l2_rate:
        # l2 regularization
        self.syn1neg[pred_index_list] -=\
            alpha * self.l2_rate * self.syn1neg[pred_index_list]
        self.syn0[feature_index_list] -=\
            alpha * self.l2_rate * self.syn0[feature_index_list]

    # weight update
    # important code snippet
    # gb: list of length is 4
    self.syn1neg[pred_index_list] += outer(gb, feature_vec)
    self.syn0[feature_index_list] +=\
        neu1e.reshape(len(feature_index_list),
                      len(neu1e) / len(feature_index_list))

    softmax_score = softmax_score[-2:]
    if softmax_score[1] > 0.5:
        label = 1
    else:
        label = 0
    prev2_label = prev_label
    prev_label = label
    if self.use_gold:
        prev_label = true_label
\end{minted}
\caption{predict\_single\_position关键代码}
\label{code:train_gold_per_sentence关键代码}
\end{listing}

\subsection{predict\_sentence\_greedy}

predict\_sentence\_greedy函数的python doc见Code~\ref{code:predict_sentence_greedy}
\begin{listing}[H]
\begin{minted}[frame=single, label=predict\_sentence\_greedy]{python}
    def predict_sentence_greedy(self, sent):
        """
        greedy predict sentence, used for test data
        :param sent: the sentence
        :return: segmented sentence, list of words
        """
\end{minted}
\caption{predict\_sentence\_greedy方法}
\label{code:predict_sentence_greedy}
\end{listing}

predict\_sentence\_greedy主要代码片段见Code~\ref{code:predict_sentence_greedy关键代码}
\begin{listing}[H]
\begin{minted}[frame=single, label=predict\_sentence\_greedy关键代码]{python}
prev2_label, prev_label = 0, 0
for p, c in enumerate(old_sentence):
    # char is still the char from original sentence,
    #  for correct eval
    if p == 0:
        target = 0
    else:
        score_list, _, _, _, _ =\
            self.predict_single_position(
                sentence, p, prev2_label,
                    prev_label, states=states)

        if self.binary_pred:
            score_list = score_list[:2]
        elif self.hybrid_pred:
            old_char = old_sentence[p]
            if old_char in self.vocab and\
                            self.vocab[old_char].count\
                            > self.hybrid_threshold:
                score_list = score_list[-2:]
            else:
                # score_list = score_list[:2]
                x, y = score_list[:2], score_list[-2:]
                score_list = [(x[i] + y[i]) / 2.0 for i in range(2)]
        else:
            score_list = score_list[-2:]

        # transform score to binary target
        if score_list[1] > 0.5:
            target = 1
        else:
            target = 0
    # update the label in the current iter
    states[p] = target

    prev2_label = prev_label
    prev_label = target
\end{minted}
\caption{predict\_sentence\_greedy关键代码}
\label{code:predict_sentence_greedy关键代码}
\end{listing}

\subsection{Call Hierarchy}

predict\_single\_position函数由
Code~\ref{cde:call_hierarchy}所列出的几个函数调用。
\begin{listing}[H]
\begin{minted}[frame=single, label=Call Hierarchy]{text}
segment_corpus(model, corpus, threashold=0) (code.predict)
Seger.train_gold_per_sentence(self, sentence, alpha, work=None) \
    (Seger in code.seg2)
Seger.predict_sentence_greedy(self, sent) (Seger in code.seg2)
\end{minted}
\caption{Call Hierarchy}
\label{cde:call_hierarchy}
\end{listing}

\section{算法改进}

\footnote{改进代码地址:\url{https://github.com/junfenglx/emws/tree/dev}}
由从左到右单向模型改为双向模型。分词时同时考虑左边和右边字的状态。

双向具体实现过程主要有两种方法可以实现:
\begin{itemize}
\item ensemble two models
\item using the labels of right characters
\end{itemize}

下面分别叙述。

\subsection{Ensemble Method}

该方式实现比较简单。就是从左向右训练出一个模型,然后从右往左训练一个模型,
将这两个模型做一个ensemble:对于每个字,将两个模型得出的两个score相加
取平均数。根据平均后的score来确定每个字的label。该方法没有改动训练过程,
仅仅是整合两个不同模型的测试结果,实现相对简单。代码实现为code文件夹下的
predict\_bidirect.py文件。
其主要实现见Code~\ref{code:bidirection_greedy_predict}

\begin{listing}
\begin{minted}[frame=single, label=bidirection greedy predict]{python}
count = 0
seg_corpus = []

for sent_no, sent in enumerate(test_corpus):
    if sent_no % 100 == 0:
        print 'num of sentence segmented:', sent_no, '...'

    tokens = []
    if sent:
        forward_scores = segment_sentence(forward_model, sent)
        back_scores = segment_sentence(back_model, sent[::-1])
        back_scores = back_scores[::-1]

        # calculates scores
        # back_scores need right shit one to align forward_scores
        scores = np.zeros_like(forward_scores)
        scores[0] = forward_scores[0]
        for i in range(1, len(forward_scores)):
            scores[i] = (forward_scores[i] + back_scores[i-1]) / 2

        for pos, score in enumerate(scores):
            if score > 0.5:
                tokens[-1] += sent[pos]
            else:
                tokens.append(sent[pos])
        count += len(sent)
        """
        print(sent)
        print(forward_scores)
        print(back_scores)
        print(' '.join(tokens))
        """
    seg_corpus.append(tokens)
\end{minted}
\caption{bidirection greedy predict}
\label{code:bidirection_greedy_predict}
\end{listing}

\pagebreak

\subsection{使用右边字状态}

增加一个配置选项:
\mint{cfg}|no_right_action_feature = False|
当其设置为False时,算法使用右边两个字的状态特征,
反之则不使用右边字状态。

由于训练时从左向右扫描汉字,所以右边两个字的当前label 无法预先
知道,因此需要寻找一种机制来获取右边两个字的label。

尝试了两种方案:
\begin{enumerate}
	\item train时,左边状态使用当前迭代,右边使用gold standard labels
	\item train时,左右全部使用gold standard labels,
	    即使用use\_gold=1开启该参数
\end{enumerate}

测试时,由于并不知道gold standard labels,
使用多次迭代来计算test sentences labels,初始时设置所有labels为1, 表示未分词,
经过多次迭代,得到test sentences 的最终labels。
关键代码实现见Code~\ref{code:iterate_greedy_predict}

\begin{listing}[H]
\begin{minted}[frame=single, label=iterate greedy predict]{python}
# initialize states vector
states = np.ones(len(sentence) + 2, dtype=np.int8)
states[0] = 0
states[-1] = 0
states[-2] = 0
if self.no_right_action_feature:
    do_greedy_predict()
else:
    for _ in range(self.iter):
        do_greedy_predict()
\end{minted}
\caption{iterate greedy predict}
\label{code:iterate_greedy_predict}
\end{listing}

第一种训练方案,测试score达到0.949, 其在第5次迭代后就达到了该水平。

第二种训练方案,测试score达到0.948。

而当使用已有模型对test的输出的labels作为右边两字的labels时(已有模型的score为0.95),
两种训练方式score都能达到0.951。这和整合两个模型取平均概率的score是一样的。

\end{document}
